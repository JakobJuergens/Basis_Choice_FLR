\documentclass[11pt,twoside,a4paper]{article}
\usepackage[hmargin=2cm, vmargin=2cm]{geometry}
\usepackage{microtype}
\usepackage{hyperref}
\usepackage[english]{babel}
\usepackage{setspace}
\usepackage{xcolor}
\usepackage[
backend=biber,
style=authoryear-comp,
]{biblatex}


\addbibresource{RMbibliography.bib}
\onehalfspacing
\parindent=0pt

\begin{document}
	\title{{\LARGE {\color{red}PLACEHOLDER-TITLE:} Functional Linear Regression in a Scalar-on-Function Setting with Applications to SOMETHING}}
	\author{Jonghun Baek, Jakob Juergens, Jonathan Willnow}
	\date{{\color{red}whenever}}
	\maketitle
	\vspace{1.5 cm}
	\begin{center}
		Research Module in Econometrics and Statistics \\
		Winter Semester 2021/2022
	\end{center}
	
	\newpage
	
	\tableofcontents
	
	\newpage
	
	\section{Introduction}
	
	\begin{itemize}
		\item Describe the idea of regressing a scalar on functional data
		\item Describing the difference to multiple linear regression intuitively
		\item Giving an intuitive example
	\end{itemize}	
	Function Data Analysis (FDA) is a relatively new field which is getting more attention as researchers from different fields collect more data from an continuous underlying process. This data still can be processed by classical statistical methods, but only FDA allows to answer questions that are tied to the information given by the smoothness of the underlying continuous process. (see \cite{levitin_introduction_2007})\\
	 As Kokoszka and Reimherr (2017) describe, FDA should be considered when one can view one or more of the variables or units of a given data set as a smooth curve or function and the interest is in analyzing samples of curves (see \cite[S.~17]{kokoszka_introduction_2017}.
	 To motivate the use of scalar-on-function regression, consider the case of a data set containing a scalar response and observations of an continuous underlying process. In economics, on application could be the regression of stock market correlations on the Global Crisis Index (GCI), where the regression allows to assess the relationship between the correlation and the GCI at every point within a window (see~\cite{Das_2019}). 

	\section{Theory}
	
	\subsection{Draft-Overview}
	\begin{itemize}
		\item Introduce the concept of random functions
		\item Introduce the concept of square integrable deterministic \& random functions
		\item Explain basis expansions (so basis of the vector space $L^2$ and b-spline basis as an example) 
		\item Motivate Karhunen-Loeve-Expansion and Eigenbasis from PCA		
		\item Explain Scalar-on-Function Regression
		\item Estimation through basis-expansion (incl. Eigenbasis) [and estimation with roughness penalty]
		\item Address approximation error due to basis-truncation
	\end{itemize}

	\subsection{Literature}
	\begin{itemize}
		\item \cite{kokoszka_introduction_2017}
		\item \cite{hsing_theoretical_2015}
		\item \cite{ramsay_functional_2005}
		\item \cite{horvath_inference_2012}
		\item \cite{cai_prediction_2006}
		\item \cite{levitin_introduction_2007}
	\end{itemize}
	
	\newpage
	\section{Simulation}
	
	\subsection{Draft-Overview}
	\begin{itemize}
		\item Motivate Simulation for some data generating process from application
		\item Describe Simulation Setting from technical standpoint (DGP, set-up for replication, ...)
		
		\item Compare estimation with \begin{enumerate}
			\item b-spline basis without addressing approximation error
			\item ... including proper treatment of approximation error
			\item Eigenbasis constructed from observations
			\end{enumerate}
	
		\item Prediction not Inference (Alternative: Focused on a testing procedure motivated by the application)
		\item Present Results
		\item Explain relevance for application
	\end{itemize}

	\subsection{Literature}
	\begin{itemize}
		\item \cite{shonkwiler_explorations_2009}
		\item R-packages: fda, refund, mgcv
	\end{itemize}
	
	\newpage
	\section{Application}

	\subsection{Draft-Overview}
	\begin{itemize}
		\item Prediction not Inference (Alternative: Focused on a testing procedure motivated by the data set)
		\item IID data set (no dependence between the curves, don't want to do functional time series)
		\item Not necessarily data from economics (like biology, sports, whatever)
		\item Smooth curves or random walk (both fine)
		\item \href{https://functionaldata.wordpress.ncsu.edu/resources/}{https://functionaldata.wordpress.ncsu.edu/resources/}
	\end{itemize}
	
	\subsection{Literature}
	\begin{itemize}
		\item \cite{carey_life_2002}
	\end{itemize}

	\section{Outlook}
	
	\subsection{Literature}
	\begin{itemize}
		\item \cite{James.2009} (shape-restrictions)
	\end{itemize}
	
	\section{Appendix}
	
	\newpage
	
	\section{Bibliography}
	\printbibliography[heading=none]	
	
\end{document}